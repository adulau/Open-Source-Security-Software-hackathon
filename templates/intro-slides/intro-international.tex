% !TEX program = XeLaTeX
\documentclass{beamer}
\usetheme[pageofpages=of,% String used between the current page and the
                         % total page count.
          bullet=circle,% Use circles instead of squares for bullets.
          titleline=true,% Show a line below the frame title.
          alternativetitlepage=true,% Use the fancy title page.
          titlepagelogo=oss-software-hackathon.png,% Logo for the first page.
%          watermark=watermark-polito,% Watermark used in every page.
%          watermarkheight=100px,% Height of the watermark.
%          watermarkheightmult=4,% The watermark image is 4 times bigger
                                % than watermarkheight.
          ]{Torino}

%\usepackage{xeCJK}
%\usepackage{ruby}
%\setCJKmainfont{Hiragino Mincho Pro}
%\renewcommand{\rubysep}{0.1ex}

\usepackage[utf8]{inputenc}
\usepackage{tikz}
\usetikzlibrary{shapes,arrows}
\usepackage{listings}
\lstset{ %
language=C,                % choose the language of the code
basicstyle=\footnotesize,       % the size of the fonts that are used for the code
numbers=left,                   % where to put the line-numbers
numberstyle=\footnotesize\color{cyan},      % the size of the fonts that are used for the line-numbers
stepnumber=1,                   % the step between two line-numbers. If it is 1 each line will be numbered
numbersep=5pt,                  % how far the line-numbers are from the code
backgroundcolor=\color{white},  % choose the background color. You must add \usepackage{color}
showspaces=false,               % show spaces adding particular underscores
showstringspaces=false,         % underline spaces within strings
showtabs=false,                 % show tabs within strings adding particular underscores
rulecolor=\color{red},
frame=single,           % adds a frame around the code
tabsize=2,          % sets default tabsize to 2 spaces
captionpos=b,           % sets the caption-position to bottom
breaklines=true,        % sets automatic line breaking
breakatwhitespace=false,    % sets if automatic breaks should only happen at whitespace
escapeinside={\%*}{*)}          % if you want to add a comment within your code
}
\usepackage{hyperref,xcolor}
\renewcommand\UrlFont{\color{red}\sffamily\tiny}

\author{\includegraphics[height=.04\paperheight]{twitter.png}@circl\_lu - \emph{TLP:WHITE}}
\title{The (potential) Art of the Hackathon}
\subtitle{A blend of ideas, code, documentation, GUI, UX to {\bf happy infosec communities}}
\institute{\url{info@hack.lu}}
\date{March 26, 2018}

\begin{document}

  \begin{frame}[t,plain]
    \titlepage
  \end{frame}

  \begin{frame}
    \frametitle{Objective}
    \begin{center}
            The objective of this hackathon is to have an interactive session in order to exchange, {\bf enhance} and discover new {\bf Open Source Security Software} and tools. The aim is not to be fully exhaustive or have a nicely packaged software at the end of the day, but to either bootstrap a new project, bootstrap enhancements on existing projects or improve {\bf interoperability}.
    \end{center}
    \begin{center}
       As we are all learning together, don't hesitate to ask questions to each other and {\bf interact during the session(s)}.
    \end{center}
  \end{frame}

  \begin{frame}
          \frametitle{What a hackathon is not (in our view) \ldots}
    \begin{itemize}
      \item \ldots CTF (Capture The Flag)
      \item \ldots Game of Code (code competition)
      \item \ldots place to measure skills or be condescending if someone does not know something
      \item \ldots place to work (Fun takes priority and everything is informal and the Hacker Ethic\footnote{\url{http://www.acrewoods.net/free-culture/the-hacker-ethic-and-meaningful-work}} prevails)
      \item \ldots sprint, Hackathons like Marathons, is not conquered by going as fast as possible early-stage, {\bf being constant and persevering is key}
    \end{itemize}
  \end{frame}

  \begin{frame}
          \frametitle{A hackathon is (in our view) a\ldots}
    \begin{itemize}
      \item \ldots place to learn from each other
      \item \ldots great opportunity to connect with like-minded people in real life
      \item \ldots constant discovery of new stuff, tools, techniques, recipes, concepts, thoughts
      \item \ldots bringing {\bf communities together\footnote{\tiny Social Architecture, Pieter Hintjens \url{http://www.foo.be/docs-free/social-architecture/main.pdf}} and improving interoperability between Open Source Security Software}
    \end{itemize}
  \end{frame}

  \begin{frame}
    \frametitle{What are requirements? Do I need to know the latest programming language?}
    \begin{itemize}
      \item Good news everyone! {\bf NO} special Ninja, Samurai or other 31337 Hax0r skillz needed.
      \item Hackathons have produced amazing: documentation, examples on software use-cases, graphic designs, etc\ldots without any magical powers ;)
      \item One requirement is: be curious, adventurous, challenging in your ideas and respectful
    \end{itemize}
  \end{frame}

  \begin{frame}
    \frametitle{HELP, I am overwhelmed \& not sure how to start}
    \begin{itemize}
      \item Relax, you are not alone, we are also overwhelmed how awesome the group is :)
      \item Look around you, amazing people are present who help each other to get started
      \item Set yourselves goals, make a quick evaluation on how realistic it is
      \item Ask and share your ideas, questions and requests
              \begin{itemize}
                      \item {\bf 5 minutes introduction per project (at the beginning of the hackathon)}
                      \item {\bf 5 minutes presentation interrupt (when ever you like)}
              \end{itemize}
    \end{itemize}
  \end{frame}

  \begin{frame}
    \frametitle{What have been done in previous OSSS hackathons? 1/3}
    \begin{itemize}
      \item MISP\footnote{\url{https://www.misp-project/}} \& Cortex\footnote{\url{https://github.com/TheHive-Project/Cortex}} integration to allow the information sharing platform MISP to connect \& use Cortex intelligence services. Cortex 1.1.1: 2-way MISP integration now a reality
      \item cve-search performed a new major release \& reorganised the contribution aspect to ease the external contribution \& test suite improvements
      \item shotovuln\footnote{\url{https://github.com/444xxk/shotovuln}} - an offensive bash script for pentesters to find generic privesc issue on Unix boxes
    \end{itemize}
  \end{frame}

  \begin{frame}
    \frametitle{What have been done in previous OSSS hackathons? 2/3}
    \begin{itemize}
      \item Viper\footnote{\url{https://github.com/viper-framework/viper}} made significant progress towards Python 3 support, including work on Python 3 port of PEfile \& the creation of an open test suite for PEfile
      \item A new project has been evaluated for the exchange of software vulnerability information within open source projects supporting software evaluation, or security assessment. The idea is to share a common format between cve-search\footnote{\url{https://www.cve-search.org/}}, aboutcode to share software vulnerabilities within open source projects
      \item Updates in JSMF-Android\footnote{\url{https://github.com/ICC-analysis/JSMF-Android}} - Analysis of Inter-Component Communication links (ICC) \& source code of Android applications (AST)
    \end{itemize}
  \end{frame}

  \begin{frame}
    \frametitle{What have been done in previous OSSS hackathons? 3/3}
    \begin{itemize}
      \item The Seeker of IoC - CERTitude\footnote{\url{https://github.com/CERT-W/certitude}} is a Python-based tool, which aims at assessing the compromised perimeter during incident response assignments
      \item Improvement of mail\_to\_misp\footnote{\url{https://github.com/MISP/mail_to_misp/}} with support for Thunderbird was added
      \item MISP taxonomy\footnote{\url{https://github.com/MISP/misp-taxonomies}} improvement with assessment of the analysts
      \item MISP galaxy\footnote{\url{https://github.com/MISP/misp-galaxy}} improved with an extended ransomware cluster
    \end{itemize}
  \end{frame}

  \begin{frame}
    \frametitle{What's next?}
    \begin{itemize}
    \item Open discussion on what ideas people already have and want to hack on
    \item Do we want to group together on certain ideas?
    \item Panic, I still do not feel comfortable on what to do\ldots $\to$ No worries, projects have idea lists\footnote{\url{https://github.com/MISP/MISP/wiki/Hackathon}}.
    \end{itemize}
  \end{frame}

  \begin{frame}
    \frametitle{Flexitime-line in Luxembourg and Japan - 26 March 2018}
    10:00 - Hackathon intro \\
    10:10 - Project 5 minutes round table\\
    12:30 - Lunch (while hacking)\\
    18:00 - Conclusions, what have we learned \\
    19:00 - The end? Next hackathon? \\
    anytime - 5 minutes break presentation
  \end{frame}


\end{document}

